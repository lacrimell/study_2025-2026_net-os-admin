\documentclass[a4paper, 12pt]{article}

\usepackage[english, russian]{babel}
\usepackage[T2A]{fontenc}
\usepackage[utf8]{inputenc}
\usepackage{indentfirst}
\usepackage{amsmath, amsfonts, amssymb, amsthm, mathtools}
\usepackage[export]{adjustbox}
\usepackage{graphicx} 
\usepackage{minted}
\AtBeginEnvironment{minted}{\setlength{\parskip}{0pt}}
\setminted{breaklines=true}

\usepackage{subcaption}
\usepackage{verbatim}
\usepackage{geometry} 
\geometry{top=25mm}
\geometry{bottom=30mm}
\geometry{left=20mm}
\geometry{right=20mm}

\linespread{1}
\setlength{\parindent}{20pt}
\setlength{\parskip}{12pt} 
\usepackage{titleps}

% \newpagestyle{main}{
% 	\setheadrule{0.4pt}
% 	\setfootrule{0.4pt}
% }
% \pagestyle{main}

\usepackage{hyperref}

\hypersetup{
    colorlinks=true,
    linkcolor=blue,
    filecolor=magenta,      
    urlcolor=black,
    pdftitle={Overleaf Example},
    pdfpagemode=FullScreen,
    }


\title{Лабораторная работа № 9. \\ Настройка POP3/IMAP сервера}
\author{Калашникова Ольга \\ НПИбд-01-23}
\institute{Российский университет дружбы народов имени Патриса Лумумбы}
\date{2025}

\begin{document}

\frame{\titlepage}

\begin{frame}
\frametitle{Цель работы}
\begin{itemize}
    \item Приобретение практических навыков по установке и простейшему конфигурированию POP3/IMAP-сервера
\end{itemize}
\end{frame}

\begin{frame}[fragile]
\frametitle{Установка Dovecot}
Установили на сервере необходимые для работы пакеты:
\begin{minted}{bash}
    sudo -i
    dnf -y install dovecot telnet
\end{minted}
\end{frame}

\begin{frame}
\frametitle{Установка Dovecot}
\centering
\includegraphics[width=\textwidth]{../images/1.png}
\captionof{figure}{Успешная установка Dovecot.}
\end{frame}

\begin{frame}
\frametitle{Настройка Dovecot - Разрешение протоколов}
\centering
\includegraphics[width=\textwidth]{../images/2.png}
\captionof{figure}{Разрешенные почтовые протоколы в конфигурации Dovecot.}
\end{frame}

\begin{frame}
\frametitle{Настройка Dovecot - Аутентификация}
\centering
\includegraphics[width=\textwidth]{../images/3.png}
\captionof{figure}{Указание метода аутентификации \texttt{plain}.}
\end{frame}

\begin{frame}
\frametitle{Настройка Dovecot - Поиск пользователей}
\centering
\includegraphics[width=\textwidth]{../images/4.png}
\captionof{figure}{Использование файла \texttt{passwd} и \texttt{pam} для поиска паролей.}
\end{frame}

\begin{frame}
\frametitle{Настройка Dovecot - Расположение почтовых ящиков}
\centering
\includegraphics[width=\textwidth]{../images/5.png}
\captionof{figure}{Настройка месторасположения почтовых ящиков пользователей.}
\end{frame}

\begin{frame}
\frametitle{Настройка Postfix}
\centering
\includegraphics[width=\textwidth]{../images/6.png}
\captionof{figure}{Создание каталога для доставки почты в Postfix.}
\end{frame}

\begin{frame}
\frametitle{Настройка межсетевого экрана}
\centering
\includegraphics[width=\textwidth]{../images/7.png}
\captionof{figure}{Настройка межсетевого экрана для работы служб протоколов POP3 и IMAP.}
\end{frame}

\begin{frame}
\frametitle{Настройка SELinux}
\centering
\includegraphics[width=\textwidth]{../images/8.png}
\captionof{figure}{Восстановление контекста безопасности в SELinux.}
\end{frame}

\begin{frame}
\frametitle{Запуск служб}
\centering
\includegraphics[width=\textwidth]{../images/9.png}
\captionof{figure}{Перезапуск Postfix и запуск Dovecot.}
\end{frame}

\begin{frame}
\frametitle{Проверка работы - Мониторинг логов}
\centering
\includegraphics[width=\textwidth]{../images/10.png}
\captionof{figure}{Запуск мониторинга работы почтовой службы.}
\end{frame}

\begin{frame}
\frametitle{Проверка работы - Просмотр почты через mail}
\centering
\includegraphics[width=\textwidth]{../images/11.png}
\captionof{figure}{Просмотр имеющейся почты на сервере.}
\end{frame}

\begin{frame}
\frametitle{Проверка работы - Просмотр через doveadm}
\centering
\includegraphics[width=\textwidth]{../images/12.png}
\captionof{figure}{Просмотр почты пользователя \texttt{oskalashnikova} через doveadm.}
\end{frame}

\begin{frame}
\frametitle{Настройка клиента - Переход в режим root}
\centering
\includegraphics[width=\textwidth]{../images/13.png}
\captionof{figure}{Переход в режим суперпользователя на виртуальной машине \texttt{client}.}
\end{frame}

\begin{frame}
\frametitle{Настройка клиента - Установка Evolution}
\centering
\includegraphics[width=\textwidth]{../images/14.png}
\captionof{figure}{Установка почтового клиента Evolution.}
\end{frame}

\begin{frame}
\frametitle{Настройка Evolution - Стартовое окно}
\centering
\includegraphics[width=0.8\textwidth]{../images/15.png}
\captionof{figure}{Стартовое окно настройки Evolution.}
\end{frame}

\begin{frame}
\frametitle{Настройка Evolution - Учетная запись}
\centering
\includegraphics[width=0.8\textwidth]{../images/16.png}
\captionof{figure}{Настройка учётной записи почты.}
\end{frame}

\begin{frame}
\frametitle{Настройка Evolution - IMAP сервер}
\centering
\includegraphics[width=0.8\textwidth]{../images/17.png}
\captionof{figure}{Настройка IMAP-сервера.}
\end{frame}

\begin{frame}
\frametitle{Настройка Evolution - SMTP сервер}
\centering
\includegraphics[width=0.8\textwidth]{../images/18.png}
\captionof{figure}{Настройка SMTP-сервера.}
\end{frame}

\begin{frame}
\frametitle{Настройка Evolution - Проверка настроек}
\centering
\includegraphics[width=0.8\textwidth]{../images/19.png}
\captionof{figure}{Проверка заданных настроек.}
\end{frame}

\begin{frame}
\frametitle{Отправка тестового письма - Окно отправки}
\centering
\includegraphics[width=\textwidth]{../images/20.png}
\captionof{figure}{Окно отправки письма.}
\end{frame}

\begin{frame}
\frametitle{Отправка тестового письма - Сохранение}
\centering
\includegraphics[width=\textwidth]{../images/21.png}
\captionof{figure}{Сообщение о том, что письмо сохранено.}
\end{frame}

\begin{frame}
\frametitle{Отправка тестового письма - Предупреждение SSL}
\centering
\includegraphics[width=\textwidth]{../images/22.png}
\captionof{figure}{Окно предупреждения о ненадёжном SSL/TLS-сертификате.}
\end{frame}

\begin{frame}
\frametitle{Отправка тестового письма - Аутентификация}
\centering
\includegraphics[width=\textwidth]{../images/23.png}
\captionof{figure}{Запрос аутентификации для почтового аккаунта.}
\end{frame}

\begin{frame}
\frametitle{Отправка тестового письма - Второе письмо}
\centering
\includegraphics[width=\textwidth]{../images/24.png}
\captionof{figure}{Отправка ещё одного письма.}
\end{frame}

\begin{frame}
\frametitle{Проверка доставки - Успешная доставка}
\centering
\includegraphics[width=\textwidth]{../images/25.png}
\captionof{figure}{Успешная доставка письма в почтовый ящик.}
\end{frame}

\begin{frame}
\frametitle{Проверка доставки - Содержание письма}
\centering
\includegraphics[width=\textwidth]{../images/26.png}
\captionof{figure}{Содержание полученного сообщения.}
\end{frame}

\begin{frame}
\frametitle{Проверка доставки - Все письма}
\centering
\includegraphics[width=\textwidth]{../images/27.png}
\captionof{figure}{Тестовые письма в почтовом ящике.}
\end{frame}

\begin{frame}
\frametitle{Мониторинг - Логи сервера}
\centering
\includegraphics[width=\textwidth]{../images/28.png}
\captionof{figure}{Просмотр логов почтовой службы при отправке писем с клиента.}
\end{frame}

\begin{frame}
\frametitle{Мониторинг - Проверка через mail}
\centering
\includegraphics[width=\textwidth]{../images/29.png}
\captionof{figure}{Просмотр полученных писем через утилиту \texttt{mail}.}
\end{frame}

\begin{frame}
\frametitle{Мониторинг - Проверка через doveadm}
\centering
\includegraphics[width=\textwidth]{../images/30.png}
\captionof{figure}{Просмотр полученных писем через утилиту \texttt{doveadm}.}
\end{frame}

\begin{frame}
\frametitle{Тестирование через Telnet}
\centering
\includegraphics[width=\textwidth]{../images/31.png}
\captionof{figure}{Подключение к почтовому серверу через протокол Telnet.}
\end{frame}

\begin{frame}
\frametitle{Настройка окружения - Копирование конфигураций}
\centering
\includegraphics[width=\textwidth]{../images/32.png}
\captionof{figure}{Копирование конфигурационных файлов Dovecot.}
\end{frame}

\begin{frame}
\frametitle{Настройка окружения - Скрипт на сервере}
\centering
\includegraphics[width=\textwidth]{../images/33.png}
\captionof{figure}{Содержание файла \texttt{mail.sh} на сервере.}
\end{frame}

\begin{frame}
\frametitle{Настройка окружения - Скрипт на клиенте}
\centering
\includegraphics[width=\textwidth]{../images/34.png}
\captionof{figure}{Содержание файла \texttt{mail.sh} на клиенте.}
\end{frame}

\begin{frame}
\frametitle{Контрольные вопросы}
\textbf{1. За что отвечает протокол SMTP?} \\
SMTP (Simple Mail Transfer Protocol) отвечает за отправку и пересылку электронной почты между серверами. Он определяет правила передачи сообщений от отправителя к получателю через промежуточные почтовые серверы.

\textbf{2. За что отвечает протокол IMAP?} \\
IMAP (Internet Message Access Protocol) позволяет клиентам получать доступ к почтовым сообщениям, хранящимся на сервере, без необходимости их скачивания на локальное устройство. Сообщения остаются на сервере, что позволяет работать с почтой с нескольких устройств.
\end{frame}

\begin{frame}
\frametitle{Контрольные вопросы}
\textbf{3. За что отвечает протокол POP3?} \\
POP3 (Post Office Protocol version 3) используется для загрузки электронной почты с сервера на локальное устройство. В отличие от IMAP, POP3 обычно удаляет сообщения с сервера после их загрузки (хотя это можно настроить).

\textbf{4. В чём назначение Dovecot?} \\
Dovecot - это open-source IMAP и POP3 сервер для Linux/UNIX-подобных систем. Его назначение - обеспечение безопасного доступа пользователей к их почтовым ящикам по протоколам IMAP и POP3.
\end{frame}

\begin{frame}
\frametitle{Контрольные вопросы}
\textbf{5. В каких файлах обычно находятся настройки работы Dovecot?} \\
Основные конфигурационные файлы Dovecot:
\begin{itemize}
\item \texttt{/etc/dovecot/dovecot.conf}  главный конфигурационный файл
\item \texttt{/etc/dovecot/conf.d/10-auth.conf}  настройки аутентификации
\item \texttt{/etc/dovecot/conf.d/auth-system.conf.ext}  системная аутентификация через PAM
\item \texttt{/etc/dovecot/conf.d/10-mail.conf}  настройки почтовых ящиков
\item \texttt{/etc/dovecot/conf.d/10-ssl.conf}  настройки SSL/TLS
\end{itemize}
\end{frame}

\begin{frame}
\frametitle{Контрольные вопросы}
\textbf{6. В чём назначение Postfix?} \\
Postfix - это MTA (Mail Transfer Agent), который отвечает за прием, обработку и доставку электронной почты. Он работает с протоколом SMTP и взаимодействует с Dovecot для предоставления полного почтового решения.

\textbf{7. Какие методы аутентификации пользователей можно использовать в Dovecot?} \\
Основные методы аутентификации в Dovecot:
\begin{itemize}
\item \texttt{plain} - обычная текстовая аутентификация (часто используется с SSL/TLS)
\item \texttt{login} - устаревший метод, похожий на plain
\item \texttt{cram-md5} - более безопасный метод с использованием challenge-response
\item \texttt{digest-md5} - аналогично cram-md5
\item \texttt{apop} - для POP3
\end{itemize}
\end{frame}

\begin{frame}[fragile]
\frametitle{Контрольные вопросы}
\textbf{8. Пример заголовка письма с пояснениями:}
\begin{minted}{text}
From: oskalashnikova@oskalashnikova.net  # Отправитель
To: oskalashnikova@oskalashnikova.net    # Получатель
Date: Tue, 15 Apr 2025 10:30:00 +0300    # Дата и время
Subject: Тестовое письмо                 # Тема
Message-ID: <20250415103000@server>      # Уникальный ID
MIME-Version: 1.0                        # Версия MIME
Content-Type: text/plain; charset=UTF-8  # Тип содержимого
\end{minted}
\end{frame}

\begin{frame}[fragile]
\frametitle{Контрольные вопросы}
\textbf{9. Примеры команд для работы с почтовыми протоколами через telnet:}
\begin{minted}{bash}
# Подключение к POP3 серверу
telnet mail.oskalashnikova.net 110
USER oskalashnikova
PASS пароль
LIST          # список писем
RETR 1        # получить первое письмо
DELE 1        # удалить первое письмо
QUIT

# Подключение к SMTP серверу
telnet mail.oskalashnikova.net 25
HELO client.oskalashnikova.net
MAIL FROM: <oskalashnikova@oskalashnikova.net>
RCPT TO: <user@example.com>
DATA          # начало ввода письма
Subject: Тест
Текст письма
.             # точка на отдельной строке завершает письмо
QUIT
\end{minted}
\end{frame}

\begin{frame}[fragile]
\frametitle{Контрольные вопросы}
\textbf{10. Примеры работы с doveadm:}
\begin{minted}{bash}
# Просмотр списка почтовых ящиков пользователя
doveadm mailbox list -u oskalashnikova

# Поиск писем по критериям
doveadm search -u oskalashnikova mailbox INBOX

# Получение информации о письме
doveadm fetch -u oskalashnikova "uid body" mailbox INBOX

# Проверка аутентификации пользователя
doveadm auth test oskalashnikova

# Просмотр статистики
doveadm stats

# Принудительное удаление блокировок
doveadm force-resync -u oskalashnikova INBOX
\end{minted}
\end{frame}

\begin{frame}
\frametitle{Выводы}
\begin{itemize}
    \item В результате выполнения лабораторной работы приобрели практические навыки по установке и простейшему конфигурированию POP3/IMAP-сервера.
\end{itemize}
\end{frame}

\end{document}
