\documentclass[a4paper, 12pt]{article}

\usepackage[english, russian]{babel}
\usepackage[T2A]{fontenc}
\usepackage[utf8]{inputenc}
\usepackage{indentfirst}
\usepackage{amsmath, amsfonts, amssymb, amsthm, mathtools}
\usepackage[export]{adjustbox}
\usepackage{graphicx} 
\usepackage{minted}
\AtBeginEnvironment{minted}{\setlength{\parskip}{0pt}}
\setminted{breaklines=true}

\usepackage{subcaption}
\usepackage{verbatim}
\usepackage{geometry} 
\geometry{top=25mm}
\geometry{bottom=30mm}
\geometry{left=20mm}
\geometry{right=20mm}

\linespread{1}
\setlength{\parindent}{20pt}
\setlength{\parskip}{12pt} 
\usepackage{titleps}

% \newpagestyle{main}{
% 	\setheadrule{0.4pt}
% 	\setfootrule{0.4pt}
% }
% \pagestyle{main}

\usepackage{hyperref}

\hypersetup{
    colorlinks=true,
    linkcolor=blue,
    filecolor=magenta,      
    urlcolor=black,
    pdftitle={Overleaf Example},
    pdfpagemode=FullScreen,
    }


\title{Отчет по лабораторной работе № 9. \\Настройка POP3/IMAP сервера}
\author{Калашникова Ольга \\ НПИбд-01-23}
\date{2025}

\begin{document}
\maketitle
\newpage

\tableofcontents

\newpage
\section{Цель работы}
Приобретение практических навыков по установке и простейшему конфигурированию POP3/IMAP-сервера.
\newpage
\section{Выполнение работы}

\subsection{Установка Dovecot}
\begin{enumerate}
\item На виртуальной машине \texttt{server} вошли под пользователем и открыли терминал. Перешли в режим суперпользователя:
  \begin{minted}{bash}
    sudo -i
  \end{minted}
\item Установили необходимые для работы пакеты(Рис. \ref{01}):
  \begin{minted}{bash}
    dnf -y install dovecot telnet
  \end{minted}
  \begin{center}
    \centering
    \includegraphics[width=\textwidth]{../images/1.png}
    \captionof{figure}{Успешная установка Dovecot.}
    \label{01}
\end{center}
\end{enumerate}

\subsection{Настройка dovecot}
\begin{enumerate}
\item В конфигурационном файле \texttt{/etc/dovecot/dovecot.conf} пропишем список почтовых протоколов, с которыми разрешено работать Dovecot (Рис. \ref{02}):
  \begin{minted}{bash}
    protocols = imap pop3
  \end{minted}
\begin{center}
    \centering
    \includegraphics[width=\textwidth]{../images/2.png}
    \captionof{figure}{Разрешенные почтовые протоколы в конфигурации Dovecot.}
    \label{02}
\end{center}

\item В конфигурационном файле \texttt{/etc/dovecot/conf.d/10-auth.conf} проверим, что указали(или уже указан) метод аутентификации \texttt{plain} (Рис. \ref{03}):
\begin{minted}{bash}
  auth_mechanisms = plain
\end{minted}
\begin{center}
    \centering
    \includegraphics[width=\textwidth]{../images/3.png}
    \captionof{figure}{Указание метода аутентификации \texttt{plain}.}
    \label{03}
\end{center}

\item В конфигурационном файле 
\texttt{/etc/dovecot/conf.d/auth-system.conf.ext} проверим, используется ли для поиска пользователей и их паролей \texttt{pam} и файл \texttt{passwd} (Рис. \ref{04}):
  \begin{minted}{bash}
    passdb {
      driver = pam
    }
    userdb {
      driver = passwd
    }
  \end{minted}
\begin{center}
    \centering
    \includegraphics[width=\textwidth]{../images/4.png}
    \captionof{figure}{Использование файла \texttt{passwd} и \texttt{pam} для поиска паролей.}
    \label{04}
\end{center}

\item В конфигурационном файле \texttt{/etc/dovecot/conf.d/10-mail.conf} настроим расположение почтовых ящиков пользователей (Рис. \ref{05}):
  \begin{minted}{bash}
    mail_location = maildir:~/Maildir
  \end{minted}
\begin{center}
    \centering
    \includegraphics[width=\textwidth]{../images/5.png}
    \captionof{figure}{Настройка месторасположения почтовых ящиков пользователей.}
    \label{05}
\end{center}

\item В Postfix зададим каталог для доставки почты(Рис. \ref{06}):
  \begin{minted}{bash}
    postconf -e 'home_mailbox = Maildir/'
  \end{minted}
  \begin{center}
    \centering
    \includegraphics[width=\textwidth]{../images/6.png}
    \captionof{figure}{Создание каталога для доставки почты.}
    \label{06}
\end{center}
\item Сконфигурируем межсетевой экран, разрешим работать службам протоколов POP3 и IMAP (Рис. \ref{07}):
  \begin{minted}{bash}
    firewall-cmd --get-services
    firewall-cmd --add-service=pop3 --permanent
    firewall-cmd --add-service=pop3s --permanent
    firewall-cmd --add-service=imap --permanent
    firewall-cmd --add-service=imaps --permanent
    firewall-cmd --reload
    firewall-cmd --list-services
  \end{minted}
\begin{center}
    \centering
    \includegraphics[width=\textwidth]{../images/7.png}
    \captionof{figure}{Настройка межсетевого экрана для работы служб протоколов POP3 и IMAP.}
    \label{07}
\end{center}

\item Восстановим контекст безопасности в SELinux (Рис. \ref{08}):
  \begin{minted}{bash}
    restorecon -vR /etc
  \end{minted}
\begin{center}
    \centering
    \includegraphics[width=\textwidth]{../images/8.png}
    \captionof{figure}{Восстановление контекста безопасности в SELinux.}
    \label{08}
\end{center}

\item Перезапустим Postfix и запустим Dovecot (Рис. \ref{09}):
  \begin{minted}{bash}
    systemctl restart postfix
    systemctl enable dovecot
    systemctl start dovecot
  \end{minted}
\begin{center}
    \centering
    \includegraphics[width=\textwidth]{../images/9.png}
    \captionof{figure}{Перезапуск Postfix и Dovecot.}
    \label{09}
\end{center}

\end{enumerate}


\subsection{Проверка работы Dovecot}
\begin{enumerate}
\item На дополнительном терминале виртуальной машины \texttt{server} запустим мониторинг работы почтовой службы (Рис. \ref{10}):
\begin{minted}{bash}
  tail -f /var/log/maillog
\end{minted}
\begin{center}
    \centering
    \includegraphics[width=\textwidth]{../images/10.png}
    \captionof{figure}{Запуск мониторинга работы почтовой службы.}
    \label{10}
\end{center}

\item На терминале сервера для просмотра имеющейся почты используем команду  (Рис. \ref{11})
  \begin{minted}{bash}
    MAIL=~/Maildir mail
  \end{minted}
\begin{center}
    \centering
    \includegraphics[width=\textwidth]{../images/11.png}
    \captionof{figure}{Просмотр имеющейся почты на сервере.}
    \label{11}
\end{center}

\item Для просмотра \texttt{mailbox} пользователя на сервере на терминале с правами суперпользователя используем (Рис. \ref{12})
  \begin{minted}{bash}
    doveadm mailbox list -u oskalashnikova
  \end{minted}
\begin{center}
    \centering
    \includegraphics[width=\textwidth]{../images/12.png}
    \captionof{figure}{Просмотр нашей почты пользователя \texttt{oskalashnikova}.}
    \label{12}
\end{center}

\item На виртуальной машине \texttt{client} войдём под пользователем и откроем терминал. Перейдём в режим суперпользователя (Рис. \ref{13}):
  \begin{minted}{bash}
    sudo -i
  \end{minted}
    \begin{center}
    \centering
    \includegraphics[width=\textwidth]{../images/13.png}
    \captionof{figure}{Переход в режим суперпользователя на виртуальной машине \texttt{client}.}
    \label{13}
\end{center}

\item Установим почтовый клиент(Рис. \ref{14})
  \begin{minted}{bash}
    dnf -y install evolution
  \end{minted}
  \begin{center}
    \centering
    \includegraphics[width=\textwidth]{../images/14.png}
    \captionof{figure}{Успешная установка evolution.}
    \label{14}
\end{center}

\item Запустим и настроим почтовый клиент Evolution (Рис. \ref{15} - \ref{19}):
  \begin{itemize}
  \item в окне настройки учётной записи почты укажем имя, адрес почты в виде \break \texttt{oskalashnikova@oskalashnikova.net}, введём пароль пользователя, нажмём <<Продолжить>>, затем <<Настроить вручную>>;
  \item в качестве IMAP-сервера для входящих сообщений и SMTP-сервера для исходящих сообщений пропишем наш почтовый адрес \texttt{mail.oskalashnikova.net}, в качестве пользователя для входящих и исходящих сообщений указали \texttt{oskalashnikova};
  \item проверим номера портов: для IMAP — порт 143, для SMTP — порт 25;
  \item проверим настройки SSL и метода аутентификации: для IMAP — STARTTLS, аутентификация по обычному паролю, для SMTP — без аутентификации, аутентификация — <<Без аутентификации>>;
  
\end{itemize}
\begin{center}
    \centering
    \includegraphics[width=\textwidth]{../images/15.png}
    \captionof{figure}{Встречающий экран Evolution.}
    \label{15}
\end{center}
\begin{center}
    \centering
    \includegraphics[width=\textwidth]{../images/16.png}
    \captionof{figure}{Настройка учётной записи (имя пользователя).}
    \label{16}
\end{center}
\begin{center}
    \centering
    \includegraphics[width=\textwidth]{../images/17.png}
    \captionof{figure}{Настройка IMAP-сервера.}
    \label{17}
\end{center}
\begin{center}
    \centering
    \includegraphics[width=\textwidth]{../images/18.png}
    \captionof{figure}{Настройка SMTP-сервера.}
    \label{17}
\end{center}
\begin{center}
    \centering
    \includegraphics[width=\textwidth]{../images/19.png}
    \captionof{figure}{Проверка заданных настроек.}
    \label{17}
\end{center}

\item Из почтового клиента отправим себе несколько тестовое письмо и убедилимся, что они доставлены  (Рис. \ref{20} - \ref{26}).
\begin{center}
    \centering
    \includegraphics[width=\textwidth]{../images/20.png}
    \captionof{figure}{Окно отправки письма.}
    \label{20}
\end{center}
\begin{center}
    \centering
    \includegraphics[width=\textwidth]{../images/21.png}
    \captionof{figure}{Сообщение о том, что письмо сохранено.}
    \label{21}
\end{center}
\begin{center}
    \centering
    \includegraphics[width=\textwidth]{../images/22.png}
    \captionof{figure}{Окно предупреждения о ненадёжном SSL/TLS-сертификате для почтового сервера.}
    \label{22}
\end{center}
\begin{center}
    \centering
    \includegraphics[width=\textwidth]{../images/23.png}
    \captionof{figure}{Запрос аутентификации для почтового аккаунта \texttt{oskalashnikova@oskalashnikova.net.}}
    \label{23}
\end{center}
\begin{center}
    \centering
    \includegraphics[width=\textwidth]{../images/24.png}
    \captionof{figure}{Отправка ещё одного письма.}
    \label{24}
\end{center}
\begin{center}
    \centering
    \includegraphics[width=\textwidth]{../images/25.png}
    \captionof{figure}{Успешная доставка в наш почтовый ящик.}
    \label{25}
\end{center}
\begin{center}
    \centering
    \includegraphics[width=\textwidth]{../images/26.png}
    \captionof{figure}{Содержание сообщения.}
    \label{26}
\end{center}

\item Для проверки направим ещё несколько тестовых писем  (Рис. \ref{27})

\begin{center}
    \centering
    \includegraphics[width=\textwidth]{../images/27.png}
    \captionof{figure}{Тестовые письма в почтовом ящике.}
    \label{27}
\end{center}

\item Параллельно посмотрим, какие сообщения выдаются при мониторинге почтовой службы на сервере, а также при использовании \texttt{doveadm} и \texttt{mail} (Рис. \ref{28}-\ref{30}).
\begin{center}
    \centering
    \includegraphics[width=\textwidth]{../images/28.png}
    \captionof{figure}{Просмотр логов почтовой службы при отправке писем с клиента.}
    \label{16}
\begin{center}
    \centering
    \includegraphics[width=\textwidth]{../images/29.png}
    \captionof{figure}{Просмотр полученных писем через утилиту \texttt{mail}.}
    \label{17}
\end{center}
\begin{center}
    \centering
    \includegraphics[width=\textwidth]{../images/30.png}
    \captionof{figure}{Просмотр полученных писем через утилиту \texttt{doveadm}.}
    \label{30}
\end{center}

\end{center}

\item Проверим работу почтовой службы, используя на сервере протокол Telnet:
  \begin{itemize}
  \item подключимся с помощью протокола Telnet к почтовому серверу по протоколу POP3 (через порт 110), введём свой логин для подключения и пароль (Рис. \ref{31}):
    \begin{minted}{bash}
      telnet mail.oskalashnikova.net 110
      user oskalashnikova
      pass ******
    \end{minted}
\begin{center}
    \centering
    \includegraphics[width=\textwidth]{../images/31.png}
    \captionof{figure}{Подключение к почтовому серверу через протокол Telnet.}
    \label{31}
\end{center}

\end{enumerate}
\subsection{Внесение изменений в настройки внутреннего окружения виртуальной машины}
\begin{enumerate}

\item На виртуальной машине \texttt{server} перейдём в каталог для внесения изменений в настройки внутреннего окружения \texttt{/vagrant/provision/server/}. В соответствующие подкаталоги поместили конфигурационные файлы Dovecot (Рис. \ref{32}):
\begin{center}
    \centering
    \includegraphics[width=\textwidth]{../images/32.png}
    \captionof{figure}{Копирование конфигурационных файлов.}
    \label{32}
\end{center}
\item Внесём изменения в файл \texttt{/vagrant/provision/server/mail.sh}, добавив в него строки (Рис. \ref{33}):
\begin{itemize}
  \item по установке Dovecot и Telnet;
  \item по настройке межсетевого экрана;
  \item по настройке Postfix в части задания месторасположения почтового ящика;
  \item по перезапуску Postfix и запуску Dovecot.
\end{itemize}
\begin{center}
    \centering
    \includegraphics[width=\textwidth]{../images/33.png}
    \captionof{figure}{Содержание файла \texttt{mail.sh}.}
    \label{32}
\end{center}
\item На виртуальной машине \texttt{client} в каталоге \texttt{/vagrant/provision/client} скорректируем файл \texttt{mail.sh}, прописав в нём (Рис. \ref{34}):
\begin{center}
    \centering
    \includegraphics[width=\textwidth]{../images/34.png}
    \captionof{figure}{Содержание файла \texttt{mail.sh} на клиенте.}
    \label{32}
\end{center}
\end{enumerate}

\section{Выводы}
В результате выполнения лабораторной работы приобрели практические навыки по установке и простейшему конфигурированию POP3/IMAP-сервера.
\end{document}
